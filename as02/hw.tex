% Set up the document
\documentclass{article}

% Page size
\usepackage[
    letterpaper,]{geometry}
\usepackage{changepage}

% Lines between paragraphs
\setlength{\parskip}{\baselineskip}
\setlength{\parindent}{0pt}

% Math
\usepackage{mathtools}
\usepackage{amssymb}
\usepackage{amsthm}
\usepackage{commath}

% Number sets
\newcommand{\C}{\mathbb{C}}
\newcommand{\N}{\mathbb{N}}
\newcommand{\Q}{\mathbb{Q}}
\newcommand{\R}{\mathbb{R}}
\newcommand{\Z}{\mathbb{Z}}

% Links
\usepackage{hyperref}

% Page numbers at top right
\usepackage{fancyhdr}
\pagestyle{fancy}
\fancyhf{}
\fancyhead[R]{\thepage}
\renewcommand\headrulewidth{0pt}

\begin{document}

\textbf{AMATH 751 assignment 2} \\
\textbf{Matt Wiens} \\
\textbf{2020-10-19}

1. Show that all solutions of the following systems exist on $[0, \infty)$:

(a)
%
\begin{align*}
    x_1^\prime &= x_2, \\
    x_2^\prime &= -x_1 - 2 x_2^3, \\
    x(0) &= x_0,
\end{align*}
%
where $x \in \R^2$.

\textit{Solution.}
Let $x(t)$ be a solution to the above system on a right-maximal
interval $J = [0, \beta)$. Let
$m(t) = \frac{1}{2} x_1^2(t) + \frac{1}{2} x_2^2(t)$. Then
%
\begin{align*}
    m^\prime(t)
        &= x_1^\prime(t) x_1(t) + x_2^\prime(t) x_2(t) \\
        &= x_2(t) x_1(t) + (-x_1(t) - 2 x_2(t)^3) x_2(t) \\
        &= - 2 x_2(t)^4 \\
        &\leq 0
\end{align*}
%
for $t \in J$, from which it follows that
%
\begin{equation*}
    \envert{x(t)}_2 = 2 m(t) \leq 2 m(0)
    = \frac{1}{2} x_{10}^2 + \frac{1}{2} x_{20}^2
\end{equation*}
%
for $t \in J$.
Thus $x(t)$ is bounded for $t \geq 0$ and by Corollary $2.3$ in the course notes, $\beta = \infty$.

\vspace{5mm}

(b)
%
\begin{align*}
    x_1^\prime &= x_1 - 2 x_1^2 - x_1 x_2, \\
    x_2^\prime &= 2 x_2 - x_2^2 - x_1 x_2, \\
    x(0) &= x_0,
\end{align*}
%
where $x \in (\R^+)^2$.

\textit{Solution.}
Again, we let $x(t)$ be a solution to the above system on a right-maximal
interval $J = [0, \beta)$. Let
$m(t) = \frac{1}{2} x_1^2(t) + \frac{1}{2} x_2^2(t)$. Then
%
\begin{align*}
    m^\prime(t)
        &= x_1^\prime(t) x_1(t) + x_2^\prime(t) x_2(t) \\
        &= (x_1(t) - 2 x_1^2(t) - x_1(t) x_2(t)) x_1(t)
            + (2 x_2(t) - x_2^2(t) - x_1(t) x_2(t)) x_2(t) \\
        &= x_1(t)^2 (1 - 2 x_1(t) - x_2(t))
            + x_2(t)^2 (2 - x_2(t) - x_1(t))
        .
\end{align*}
%
for $t \in J$.
Note that $m^\prime(t) \leq 0$ whenever $1 - 2 x_1(t) - x_2(t) \leq 0$
and $2 - x_2(t) - x_1(t) \leq 0$. Hence, we have two cases, either
$1 - 2 x_1(t) - x_2(t) \geq 0$
or $2 - x_2(t) - x_1(t) \geq 0$ are satisfied, in which case $x(t)$ is clearly bounded, or
both $1 - 2 x_1(t) - x_2(t) \leq 0$ and $2 - x_2(t) - x_1(t) \leq 0$ and thus
$|x(t)|_2^\prime = 2 m^\prime(t) \leq 0$. In either case, it follows that $x(t)$ is bounded and thus
by Corollary $2.3$ in the course notes, $\beta = \infty$.

\newpage

2. Let $f \in C(\R^+ \times \R^n, \R^n)$, and let $x(t)$ be a solution of IVP
%
\begin{equation*}
    x^\prime(t) = f(t, x), \quad x(t_0) = x_0
\end{equation*}
%
on a right-maximal interval $J = [t_0, \beta)$. Show that either $\beta = \infty$
or
%
\begin{equation*}
    \limsup_{t \to \beta^-} |x(t)| = \infty.
\end{equation*}

\textit{Solution.}
If $J$ is a right-maximal interval than either $\beta$ is infinite or finite.
If $\beta$ is finite, suppose for contradiction that
%
\begin{equation*}
    \limsup_{t \to \beta^-} |x(t)| < \infty.
\end{equation*}
%
If we restrict our aim to some open bounded subset $D \subset \R^+ \times \R^n$ where
$(t, x(t)) \in D$ for all $t \in [c, \beta)$, for some $c \in [t_0, \beta)$.
Then since $f$ is bounded on $\overline{D}$ (because both $f$ is continuous and our above
assumption means it is bounded on $\partial D$), by Lemma $2.2$ of the course notes,
$(\beta, x(\beta))$ exists and is finite, and hence
we can extend the solution to an interval $[t_0, \beta + \alpha)$ for some
$\alpha > 0$ using Corollary $2.2$ from the course notes. However this contradicts the assumption
that $J$ was right-maximal, and so we must have
%
\begin{equation*}
    \limsup_{t \to \beta^-} |x(t)| = \infty
\end{equation*}
%
if $\beta$ is finite.

\newpage

3. Let $J = (a, b]$ and let $m, v \in C(J, R)$ such that $v(t) \geq 0$ for
$t \in J$ and for some $c \in \R$,
%
\begin{equation*}
    m(t) \leq c + \int_t^b v(s) m(s) \dif s
\end{equation*}
%
for $t \in J$. Show that for $t \in J$,
%
\begin{equation*}
    m(t) \leq c \exp\del{\int_t^b v(s) \dif s}.
\end{equation*}

\textit{Solution.}
%
Here we let, where $t \in J$ (here and throughout the rest of the proof),
%
\begin{equation*}
    w(t) \coloneqq c + \int_t^b v(s) m(s) \dif s
         = c - \int_b^t v(s) m(s) \dif s
    .
\end{equation*}
%
Taking a derivative, we have
%
\begin{equation*}
    w^\prime(t) = - v(t) m(t) \geq - v(t) w(t)
\end{equation*}
%
and so
%
\begin{equation*}
    w^\prime(t) + v(t) w(t) \geq 0
    .
\end{equation*}
%
Multiplying through by the integrating factor
%
\begin{equation*}
    \exp\del{\int_b^t v(s) \dif s},
\end{equation*}
%
we have
%
\begin{equation*}
    \dod{}{t} \sbr{
        w(t)
        \exp\del{\int_b^t v(s) \dif s}
    }
    \geq 0
    .
\end{equation*}
%
Thus,
%
\begin{equation*}
    w(t) \exp\del{\int_b^t v(s) \dif s} \leq w(b) = c
\end{equation*}
%
and thus
%
\begin{equation*}
    m(t)
    \leq w(t)
    \leq c \exp\del{- \int_b^t v(s) \dif s}
    = c \exp\del{\int_t^b v(s) \dif s}
    .
\end{equation*}

\newpage

4. Let $g \in C(\R, \R^n)$ and $A(t)$ be an $n \times n$ matrix of continuous functions on $\R$.
Show that the IVP
%
\begin{equation*}
    x^\prime(t) = A(t) x + g(t), \quad x(t_0) = x_0
\end{equation*}
%
has a unique solution in $(-\infty, \infty)$.

\textit{Solution.}
We can write the above system as
%
\begin{equation*}
    x^\prime(t) = f(t, x), \quad x(t_0) = x_0
    ,
\end{equation*}
%
where
%
\begin{equation*}
    f(t, x) = A(t) x + g(t)
    .
\end{equation*}
%
Hence
%
\begin{equation*}
    |f(t, x)|
    = |A(t) x + g(t)|
    \leq |A(t) x| + |g(t)|
    \leq \enVert{A(t)} |x| + |g(t)|
    .
\end{equation*}
%
Noting that $\enVert{A(t)} \in C(\R, \R^+)$, then taking
$a(t) = \enVert{A(t)}$, $b(t) = |g(t)|$, by Theorem $2.4$ in the course
notes we have that the IVP has a solution $x(t)$ on $\R$.

To show that this solution is unique, suppose that two $x(t)$ and $x^*(t)$ satisfy the above system.
Noting that if $m(t) = |x(t) - x^*(t)|$, $c = 0$, and $v(t) = \enVert{A(t)}$,
then
%
\begin{equation*}
    m(t) \leq c + \int_{t_0}^t v(s) m(s) \dif s
\end{equation*}
%
for $t \in \R$ (this follows from what we showed above). Then we can apply Lemma $2.3$ from the
course notes to get that
%
\begin{equation*}
    m(t) \leq 0
\end{equation*}
%
for $t \in \R$ and so
%
\begin{equation*}
    |x(t) - x^*(t)| = 0
\end{equation*}
%
for all $t \in \R$, from which it follows that $x(t) \equiv x^*(t)$, which
proves uniqueness.

\newpage

5. Consider the system
%
\begin{equation}
    x^\prime(t) = f(x)
    \label{eq:5-1}
\end{equation}
%
where $f \in C^1(\Omega, \R^2)$ and $\Omega \subset \R^2$ is an open set. Suppose that~\eqref{eq:5-1}
has a periodic solution $\gamma$ which encloses a set $K \subset \Omega$. Show that all solutions
of~\eqref{eq:5-1} with $x(t_0) = x_0 \in K$ exist on $(- \infty, \infty)$.

\textit{Solution.}
Note that~\eqref{eq:5-1} is an autonomous system, and so its distinct
solutions never intersect (this was showed on the last assignment).
In our case, this means that any solutions that start within $K$ must
remain within $K$ for all times $t$. Hence all solutions with $x(t_0) = x_0$
exist on $(-\infty, \infty)$.

\newpage

6. Let $f \in C(Q, \R^n)$ with $\Q \subset \R^{n + 1}$ an open set. Let $U$ be a compact set and
$V$ be a bounded open set such that $U \subset V$ and $\overline{V} \subset Q$. Show that for any
$(t_0, x_0) \in U$, the IVP
%
\begin{equation*}
    x^\prime(t) = f(t, x), \quad x(t_0) = x_0
\end{equation*}
%
has a solution $x(t)$ defined on $[t_1, t_2]$ such that $(t_i, x(t_i)) \not\in U$ for $i = 1, 2$.

\textit{Solution.}
According to Corollary $2.2$ from the course notes, there exists some $\alpha_1 > 0$ such that
the IVP has a solution on $[t_0 - \alpha_1, t_0 + \alpha_1]$. We can continue this procedure to extend on the right to
$[t_0 - \alpha_1, t_0 + \alpha_2]$ where $\alpha_2 > \alpha_1$. Since $U$ is a compact set, as discussed in lectures,
after a finite number of extensions $n$ we can extend the solution to
$[t_0 - \alpha_1, t_0 + \alpha_n]$, such that $\alpha_1 < \alpha_2 < \cdots < \alpha_n$ and
$(t_0 + \alpha_n, x(t_0 + \alpha_n)) \not\in U$. Repeating this procedure on the right, we can find a finite sequence
$\alpha_1 < \beta_2 < \cdots < \beta_m$ such that the solution exists on $[t_0 - \beta_m, t_0 + \alpha_n]$ where
$(t_0 - \beta_m, x(t_0 - \beta_m)) \not\in U$. Hence identifying
$t_1 \coloneqq t_0 - \beta_m, t_2 \coloneqq t_0 + \alpha_n$, we have shown the result.

\newpage

7. Let $W \in C^1(\R^n, \R)$ such that
%
\begin{equation*}
    \lim_{|x| \to \infty} W(x) = \infty
    .
\end{equation*}
%
Suppose $f \in C(\R^+ \times \R^n, \R^n)$ and for some $M > 0$,
%
\begin{equation*}
    \dod{W(x)}{x} f(t, x) \leq 0
\end{equation*}
%
for $|x| \geq M$ and $t \in \R^+$. Show that all solutions of the IVP
%
\begin{equation*}
    x^\prime(t) = f(t, x), \quad x(t_0) = x_0
\end{equation*}
%
exist on $[t_0, \infty)$.

\textit{Solution.}
hey

\end{document}
