% Set up the document
\documentclass{article}

% Page size
\usepackage[
    letterpaper,]{geometry}
\usepackage{changepage}

% Lines between paragraphs
\setlength{\parskip}{\baselineskip}
\setlength{\parindent}{0pt}

% Math
\usepackage{mathtools}
\usepackage{amssymb}
\usepackage{amsthm}
\usepackage{commath}

% Number sets
\newcommand{\C}{\mathbb{C}}
\newcommand{\N}{\mathbb{N}}
\newcommand{\Q}{\mathbb{Q}}
\newcommand{\R}{\mathbb{R}}
\newcommand{\Z}{\mathbb{Z}}

% Links
\usepackage{hyperref}

% Page numbers at top right
\usepackage{fancyhdr}
\pagestyle{fancy}
\fancyhf{}
\fancyhead[R]{\thepage}
\renewcommand\headrulewidth{0pt}

\begin{document}

\textbf{AMATH 751 assignment 3} \\
\textbf{Matt Wiens} \\
\textbf{2020-11-02}

1. Let $\Phi(t, t_0)$ be the fundamental solution of the linear system
%
\begin{equation*}
    y^\prime(t) = A(t) y(t),
\end{equation*}
%
where $A(t)$ is an $n \times n$ matrix of continuous functions in $\R$. Show that

(i) $\Phi(t, t_2) \Phi(t_2, t_1) = \Phi(t, t_1)$, for all $t_1, t_2 \in \R$.

\textit{Solution.}
Let $c \in \R^n$ be an arbitrary column vector.
Note that both
%
\begin{equation*}
    x(t) \coloneqq \Phi(t, t_1) c \in C(\R, \R^n)
\end{equation*}
%
and
%
\begin{equation*}
    \tilde{x}(t) \coloneqq \Phi(t, t_2) x(t_2) = \Phi(t, t_2) \Phi(t_2, t_1) c \in C(\R, \R^n)
\end{equation*}
%
are solutions to the IVP
%
\begin{equation*}
    y^\prime(t) = A(t)y(t), \quad y(t_2) = \Phi(t_2, t_1) c
    .
\end{equation*}
%
Note that the initial condition is satisfied by $\tilde{x}(t)$
due to the fact that $\Phi(t_2, t_2) = I$.
By Corollary $2.7$ of the course notes, the solution to the
IVP must be unique, and so we have
%
\begin{equation*}
    x(t) \equiv \tilde{x}(t)
\end{equation*}
%
on $\R$, and so
%
\begin{equation*}
    \Phi(t, t_1) c = \Phi(t, t_2) \Phi(t_2, t_1) c
\end{equation*}
%
for all $c \in \R^n$. Thus
%
\begin{equation*}
    \Phi(t, t_1) = \Phi(t, t_2) \Phi(t_2, t_1)
\end{equation*}
%
for all $t_1, t_2 \in \R$.

\vspace{5mm}

(ii) $\Phi^{-1}(t, t_0)$ exists and $\Phi^{-1}(t, t_0) = \Phi(t_0, t)$.

\textit{Solution.}
From property (i) we have that
$\Phi(t, t_1) \Phi(t_1, t_0) = \Phi(t, t_0)$, for all $t_0, t_1 \in \R$.
Hence it follows that by substituting in $t_0$ for $t$ and using the fact that
$\Phi(t_0, t_0) = I$ that
%
\begin{equation*}
    \Phi(t_0, t_1) \Phi(t_1, t_0) = \Phi(t_0, t_0) = I
    .
\end{equation*}
%
Hence $\Phi^{-1}(t_0, t_1)$ exists and is given by $\Phi^{-1}(t_0, t_1) = \Phi(t_1, t_0)$.
Since $t_0$ and $t_1$ were arbitrary, property (ii) follows.

\vspace{5mm}

(iii) The trivial solution is asymptotically stable if
$\enVert[0]{\Phi(t, t_0)} \to 0$ as $t \to \infty$.

\textit{Solution.}
hey

\newpage

2. Show that the IVP
%
\begin{equation*}
    h^\prime(t) = - \alpha \sqrt{h(t)}, \quad h(0) = 0, \quad \alpha > 0,
\end{equation*}
%
has infinitely many different solutions. Find the maximal and minimal solutions.

\textit{Solution.}
hey

\newpage

3. Find all the equilibria, discuss their stability properties, and sketch the phase
portraits of the system
%
\begin{align*}
    x_1^\prime(t) &= x_1(t) (2 - 3 x_1(t) + x_2(t)), \\
    x_2^\prime(t) &= x_2(t) (1 + 2 x_1(t) - 3 x_2(t)).
\end{align*}

\textit{Solution.}
hey

\newpage

4. Consider the system
%
\begin{align*}
    x_1^\prime(t) &= - [x_1(t)]^3 + x_2(t), \\
    x_2^\prime(t) &= -\frac{1}{2} x_1(t) - x_2(t) + d \sin t,
\end{align*}
%
where $d$ is a constant.

(a) Show that all solutions exist on $[0, \infty)$.

\textit{Solution.}
hey

\vspace{5mm}

(b) If $d = 0$, show that the trivial solution is asymptotically stable.

\textit{Solution.}
hey

\newpage

5. Determine the stability property of the trivial solution of the system
%
\begin{align*}
    x_1^\prime(t) &= -2t x_1(t) + 2 [x_1(t)]^3 x_2(t) \sin^2 t, \\
    x_2^\prime(t) &= -t x_2(t) + 3 x_1(t) [x_2(t)]^7 e^{-t}.
\end{align*}

\textit{Solution.}
hey

\newpage

6. Consider the system
%
\begin{align*}
    x^\prime(t) &= y(t)^3 - 4 x(t), \\
    y^\prime(t) &= y(t)^3 - y(t) - 3 x(t).
\end{align*}
%
(a) Find all the equilibria and discuss their stability properties.

\textit{Solution.}
hey

\vspace{5mm}

(b) Show that the line $y = x$ is invariant under the flow of the system.

\textit{Solution.}
hey

\vspace{5mm}

(c) Show that $|y(t) - x(t)| \to 0$ as $t \to \infty$.

\textit{Solution.}
there's a hint to find a differential equation for $y - x$

\newpage

7. Let $f \in C(\R^+ \times \R^n, \R^n)$ and $g \in C(\R^+ \times \R, \R^+)$.
Assume that (i) $g(t, u)$ is nondecreasing in $u$ for each fixed $t$ and
%
\begin{equation*}
    |f(t, x)| \leq g(t, |x|), \quad (t, x) \in \R^+ \times \R^n;
\end{equation*}
%
(ii) every solution $u(t) = u(t, t_0, u_0)$ of the IVP
%
\begin{equation*}
    u^\prime(t) = g(t, u(t)), \quad u(t_0) = u_0
\end{equation*}
%
is bounded.

Show that every solution of the IVP
%
\begin{equation*}
    x^\prime(t) = f(t, x(t)), \quad x(t_0) = x_0
\end{equation*}
%
exists on $[t_0, \infty)$ and converges to a constant vector as $t \to \infty$.

\textit{Solution.}
hey

\end{document}
