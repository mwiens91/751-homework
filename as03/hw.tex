% Set up the document
\documentclass{article}

% Page size
\usepackage[
    letterpaper,]{geometry}
\usepackage{changepage}

% Lines between paragraphs
\setlength{\parskip}{\baselineskip}
\setlength{\parindent}{0pt}

% Math
\usepackage{mathtools}
\usepackage{amssymb}
\usepackage{amsthm}
\usepackage{commath}

% Number sets
\newcommand{\C}{\mathbb{C}}
\newcommand{\N}{\mathbb{N}}
\newcommand{\Q}{\mathbb{Q}}
\newcommand{\R}{\mathbb{R}}
\newcommand{\Z}{\mathbb{Z}}

% Links
\usepackage{hyperref}

% Page numbers at top right
\usepackage{fancyhdr}
\pagestyle{fancy}
\fancyhf{}
\fancyhead[R]{\thepage}
\renewcommand\headrulewidth{0pt}

\begin{document}

\textbf{AMATH 751 assignment 3} \\
\textbf{Matt Wiens} \\
\textbf{2020-11-11}

1. Let $\Phi(t, t_0)$ be the fundamental solution of the linear system
%
\begin{equation*}
    y^\prime(t) = A(t) y(t),
\end{equation*}
%
where $A(t)$ is an $n \times n$ matrix of continuous functions in $\R$. Show that

(i) $\Phi(t, t_2) \Phi(t_2, t_1) = \Phi(t, t_1)$, for all $t_1, t_2 \in \R$.

\textit{Solution.}
Let $c \in \R^n$ be an arbitrary column vector.
Note that both
%
\begin{equation*}
    x(t) \coloneqq \Phi(t, t_1) c \in C(\R, \R^n)
\end{equation*}
%
and
%
\begin{equation*}
    \tilde{x}(t) \coloneqq \Phi(t, t_2) x(t_2) = \Phi(t, t_2) \Phi(t_2, t_1) c \in C(\R, \R^n)
\end{equation*}
%
are solutions to the IVP
%
\begin{equation*}
    y^\prime(t) = A(t)y(t), \quad y(t_2) = \Phi(t_2, t_1) c
    .
\end{equation*}
%
Note that the initial condition is satisfied by $\tilde{x}(t)$
due to the fact that $\Phi(t_2, t_2) = I$.
By Corollary $2.7$ of the course notes, the solution to the
IVP must be unique, and so we have
%
\begin{equation*}
    x(t) \equiv \tilde{x}(t)
\end{equation*}
%
on $\R$, and so
%
\begin{equation*}
    \Phi(t, t_1) c = \Phi(t, t_2) \Phi(t_2, t_1) c
\end{equation*}
%
for all $c \in \R^n$. Thus
%
\begin{equation*}
    \Phi(t, t_1) = \Phi(t, t_2) \Phi(t_2, t_1)
\end{equation*}
%
for all $t_1, t_2 \in \R$.

\vspace{5mm}

(ii) $\Phi^{-1}(t, t_0)$ exists and $\Phi^{-1}(t, t_0) = \Phi(t_0, t)$.

\textit{Solution.}
From property (i) we have that
$\Phi(t, t_1) \Phi(t_1, t_0) = \Phi(t, t_0)$, for all $t_0, t_1 \in \R$.
Hence it follows that by substituting in $t_0$ for $t$ and using the fact that
$\Phi(t_0, t_0) = I$ that
%
\begin{equation*}
    \Phi(t_0, t_1) \Phi(t_1, t_0) = \Phi(t_0, t_0) = I
    .
\end{equation*}
%
Hence $\Phi^{-1}(t_0, t_1)$ exists and is given by $\Phi^{-1}(t_0, t_1) = \Phi(t_1, t_0)$.
Since $t_0$ and $t_1$ were arbitrary, property (ii) follows.

\vspace{5mm}

(iii) The trivial solution is asymptotically stable if
$\enVert[0]{\Phi(t, t_0)} \to 0$ as $t \to \infty$.

\textit{Solution.}
Note that the solution to the linear system is
%
\begin{equation*}
    y(t) \coloneqq y_0 \Phi(t, t_0)
    .
\end{equation*}
%
Given that $\enVert[0]{\Phi(t, t_0)} \to 0$ as $t \to \infty$, for any $\epsilon > 0$,
there exists $t^*$ such that $\enVert[0]{\Phi(t, t_0)} < \epsilon$ for all $t \geq t^*$.
Let
%
\begin{equation*}
    k = \max\cbr{\max_{t_0 \leq t < t^*} \enVert[0]{\Phi(t, t_0)}, \epsilon}
    .
\end{equation*}
%
Then if we let
%
\begin{equation*}
    |y_0| \leq \frac{\epsilon}{k}
\end{equation*}
%
then we will have $|y(t)| < \epsilon$ for all $t \geq t_0$.
Hence the trivial solution is S1.

To show that the trivial solution is S3, we have that for any $|y_0|$,
$|y(t)| \to 0$ as $t \to \infty$.

Hence, the trivial solution is asymptotically stable since it is S1 and S3.

\newpage

2. Show that the IVP
%
\begin{equation*}
    h^\prime(t) = - \alpha \sqrt{h(t)}, \quad h(0) = 0, \quad \alpha > 0,
\end{equation*}
%
has infinitely many different solutions. Find the maximal and minimal solutions.

\textit{Solution.}
To show that this IVP has infinitely many solutions, we can easily see that for all $t^* \leq 0$
%
\begin{equation*}
    h_{t^*}(t) =
    \begin{dcases}
        \frac{\alpha^2}{4} (t - t^*)^2, & 0 < t^* \leq 0, \\
        0, & t > t^*
    \end{dcases}
\end{equation*}
%
solves the IVP. Given that all solutions of the IVP are non-negative, we can see straight away
that the zero solution $h(t) \equiv 0$ is the minimal solution of the
IVP (this corresponds to taking the limit $t^* \to - \infty$ in
$h_{t^*}(t)$). Using the definition of a maximal solution,
for all $t^*$ we have that
%
\begin{equation*}
    \lim_{t \to 0^+} (h_0(t) - h_{t^*}(t)) = \lim_{t \to 0^+} h_0(t) \geq 0
    ,
\end{equation*}
%
and so $h_0(t)$ is a maximal solution of the IVP.

\newpage

3. Find all the equilibria, discuss their stability properties, and sketch the phase
portraits of the system
%
\begin{align*}
    x_1^\prime(t) &= x_1(t) (2 - 3 x_1(t) + x_2(t)), \\
    x_2^\prime(t) &= x_2(t) (1 + 2 x_1(t) - 3 x_2(t)).
\end{align*}

\textit{Solution.}
To find all equilibria, we solve
%
\begin{align*}
    0 &= x_1(t) (2 - 3 x_1(t) + x_2(t)), \\
    0 &= x_2(t) (1 + 2 x_1(t) - 3 x_2(t)).
\end{align*}
%
which has solutions
%
\begin{equation*}
    (x_1, x_2) \in \cbr{(0, 0), \del{0, \frac{1}{3}}, \del{\frac{2}{3}, 0}, (1, 1)}
    .
\end{equation*}
%
Now, we have that
%
\begin{equation*}
    Df =
    \begin{pmatrix}
        2 - 6 x_1 + x_2 & x_1 \\
        2 x_2 & 1 + 2 x_1 - 6 x_2
    \end{pmatrix}
    .
\end{equation*}
%
Evaluating $Df$ at each of the equilibrium points we have
%
\begin{align*}
    Df(0, 0) &=
    \begin{pmatrix}
        2 & 0 \\
        0 & 1
    \end{pmatrix}
    ,
    \\
    Df\del{0, \frac{1}{3}} &=
    \begin{pmatrix}
        \frac{7}{3} & 0 \\
        \frac{2}{3} & -1
    \end{pmatrix}
    ,
    \\
    Df\del{\frac{2}{3}, 0} &=
    \begin{pmatrix}
        -2 & \frac{2}{3} \\
        0 & \frac{7}{3}
    \end{pmatrix}
    ,
    \\
    Df(1, 1) &=
    \begin{pmatrix}
        - 3 & 1 \\
        2 & - 3
    \end{pmatrix}
    .
\end{align*}
%
These matrices have eigenvalues (in the order in which they appear above)
%
\begin{align*}
    \lambda &= 2, 1, \\
    \lambda &= \frac{7}{3}, -1, \\
    \lambda &= \frac{7}{3}, -2, \\
    \lambda &= -3 \pm \sqrt{2}.
\end{align*}
%
Thus we see that $(0, 0)$ is a nonlinear source and so is not stable; both
$(0, 1/3)$ and $(2/3, 0)$ are saddles and so again these are not stable;
however we see that $(1, 1)$ is a nonlinear sink with both eigenvalues having
negative real part, and so is asymptotically stable.

\newpage

4. Consider the system
%
\begin{align*}
    x_1^\prime(t) &= - [x_1(t)]^3 + x_2(t), \\
    x_2^\prime(t) &= -\frac{1}{2} x_1(t) - x_2(t) + d \sin t,
\end{align*}
%
where $d$ is a constant.

(a) Show that all solutions exist on $[0, \infty)$.

\textit{Solution.}
To show that all solutions exist on $[0, \infty)$ we can to show that $|x(t)|_2$ is bounded
on $[0, c)$ for any $c > 0$. Let $V(t) = \frac{1}{2} x_1^2 + x_2^2$.
Then
%
\begin{align*}
    V^\prime(t)
        &= x_1(t) x_1^\prime(t) + 2 x_2(t) x_2^\prime(t) \\
        &= x_1(t) \del{- [x_1(t)]^3 + x_2(t)} + 2 x_2(t) \del{-\frac{1}{2} x_1(t) - x_2(t) + d \sin t} \\
        &= - [x_1(t)]^4 + x_1(t) x_2(t) - x_1(t)x_2(t) - 2 x_2^2(t) + 2 d x_2(t) \sin t \\
        &= - [x_1(t)]^4 - 2 x_2^2(t) + 2 d x_2(t) \sin t \\
        &= - [x_1(t)]^4 - 2 x_2(t) (x_2(t) - d \sin t)
        .
\end{align*}
%
Note that this is non-positive whenever both $x_2(t) \geq 0$ and $x_2 \geq d \sin(t)$ or more simply $x_2 \geq |d|$;
or when $x_2(t) \leq 0$ and $x_2(t) \leq d \sin(t)$ or more simply $x_2(t) \leq - |d|$. Hence it can only positive
provided that $-|d| < x_2(t) < |d|$ or more simply when $|x_2(t)| < |d|$. However, we can also see that even if $|x_2(t)| < |d|$,
$|x_1(t)|$ is clearly bounded by some constant $k > 0$ for $V^\prime(t)$ to remain positive.

Noticing that $|x(t)|_2 \leq 2 V(t)$, we have that whenever $|x(t)|_2 \geq |(k, |d|)|_2 \eqqcolon r$, $V^\prime(t) \leq 0$,
and thus we have that for all $c > 0$,
%
\begin{equation*}
    \max_{0 \leq t \leq c} |x(t)|_2 = \max \cbr{|x(0)|, r}
    .
\end{equation*}
%
Hence $|x(t)|_2$ is bounded on $[0, c)$ for all $c > 0$ and thus all solutions exist on $[0, \infty)$.

\vspace{5mm}

(b) If $d = 0$, show that the trivial solution is asymptotically stable.

\textit{Solution.}
If $d = 0$, then we have that, from part (a),
%
\begin{equation*}
    \dod{|x(t)|_2}{t} \leq 2 V^\prime(t) = - [x_1(t)]^4 - 2 [x_2(t)]^2 \leq 0
        .
\end{equation*}
%
and hence as $t \to \infty$, $|x(t)|_2 \to 0$ and so $x(t) \to 0$. Clearly the trivial solution is S3.
To show that it is S1, for all $\epsilon > 0$ if $|x_0| < \epsilon$ then since the derivative of the norm
is strictly non-increasing, we have that $|x(t)| < \epsilon$ for all $t \geq t_0$. Thus the trivial solution is
S1 and S3 and so it is asymptotically stable.

\newpage

5. Determine the stability property of the trivial solution of the system
%
\begin{align*}
    x_1^\prime(t) &= -2t x_1(t) + 2 [x_1(t)]^3 x_2(t) \sin^2 t, \\
    x_2^\prime(t) &= -t x_2(t) + 3 x_1(t) [x_2(t)]^7 e^{-t}.
\end{align*}

\textit{Solution.}
Let us write this system as
%
\begin{equation*}
    \begin{pmatrix}
        x_1^\prime \\
        x_2^\prime
    \end{pmatrix}
    =
    \begin{pmatrix}
        -2t & 0 \\
        0 & -t
    \end{pmatrix}
    \begin{pmatrix}
        x_1 \\
        x_2
    \end{pmatrix}
    +
    \begin{pmatrix}
        2 [x_1(t)]^3 x_2(t) \sin^2 t \\
        3 x_1(t) [x_2(t)]^7 e^{-t}
    \end{pmatrix}
    .
\end{equation*}
%
Using the notation in the course notes, we identify
%
\begin{align*}
    A(t) &=
    \begin{pmatrix}
        -2t & 0 \\
        0 & -t
    \end{pmatrix},
    \\
    g(t, x) &=
    \begin{pmatrix}
        2 [x_1(t)]^3 x_2(t) \sin^2 t \\
        3 x_1(t) [x_2(t)]^7 e^{-t}
    \end{pmatrix}
    .
\end{align*}
%
Just by inspection we can see that the fundamental solution $\Phi(t, 0)$
is given by
%
\begin{equation*}
    \Phi(t, 0) =
    \begin{pmatrix}
        e^{-2t} & 0 \\
        0 & e^{-t}
    \end{pmatrix}
\end{equation*}
%
and hence
%
\begin{equation*}
    \enVert[0]{\Phi(t, 0)}_2 = e^{-t}
    .
\end{equation*}
%
Also, we have that
%
\begin{align*}
    |g(t, x)|^2 &=
        (2 x_1^3 x_2 \sin^2 t)^2
        +
        (3 x_1 x_2^7 e^{-t})^2
        \\
        &=
        4 x_1^6 x_2^2 \sin^4 t
        +
        9 x_1^2 x_2^{14} e^{-2t}
\end{align*}
%
When $|x| < \frac{1}{10}$, we clearly have that $|g(t, x)| \leq |x|$ for all $t \in \R^+$.
Hence by Theorem $3.6$ in the course notes, the trivial solution is uniformly asymptotically stable.

\newpage

6. Consider the system
%
\begin{align*}
    x^\prime(t) &= y(t)^3 - 4 x(t), \\
    y^\prime(t) &= y(t)^3 - y(t) - 3 x(t).
\end{align*}
%
(a) Find all the equilibria and discuss their stability properties.

\textit{Solution.}
To find all equilibria, we solve
%
\begin{align*}
    0 &= y(t)^3 - 4 x(t), \\
    0 &= y(t)^3 - y(t) - 3 x(t).
\end{align*}
%
which has solutions
%
\begin{equation*}
    (x, y) \in \cbr{(0, 0), (2, 2), (-2, -2)}
    .
\end{equation*}
%
Given that
%
\begin{equation*}
    Df =
    \begin{pmatrix}
        -4 & 3 y^2 \\
        -3 & 3 y^2 - 1
    \end{pmatrix}
    .
\end{equation*}
%
Evaluating $Df$ at each of the equilibrium points we have
%
\begin{align*}
    Df(0, 0) &=
    \begin{pmatrix}
        -4 & 0 \\
        -3 & -1
    \end{pmatrix}
    ,
    \\
    Df(2, 2) &=
    \begin{pmatrix}
        -4 & 12 \\
        -3 & 11
    \end{pmatrix}
    = Df(-2, -2)
    .
\end{align*}
%
These matrices have eigenvalues (in the order in which they appear above)
%
\begin{align*}
    \lambda &= -1, -4, \\
    \lambda &= 8, -1
    .
\end{align*}
%
Thus we see that $(0, 0)$ is a nonlinear sink with both eigenvalues having
negative real parts and so is asymptotically stable, and that both
$(2, 2)$ and $(-2, -2)$ are saddles and so are unstable.

\vspace{5mm}

(b) Show that the line $y = x$ is invariant under the flow of the system.

\textit{Solution.}
Notice that
%
\begin{equation*}
    \frac{y^\prime}{x^\prime} = \frac{y^3 - 4x}{y^3 - y - 3x}
    ,
\end{equation*}
%
and so if $y = x$ then
%
\begin{equation*}
    \frac{y^\prime}{x^\prime} = \frac{x^3 - 4x}{x^3 - x - 3x} = \frac{x^3 - 4x}{x^3 - 4x} = 1
    ,
\end{equation*}
%
which shows that the line is invariant under the flow of the system.

\vspace{5mm}

(c) Show that $|y(t) - x(t)| \to 0$ as $t \to \infty$.

\textit{Solution.}
Note that from our equations above,
%
\begin{equation*}
    \dod{(y - x)}{t} = (y^3 - y - 3x) - (y^3 - 4x) = - (y - x)
\end{equation*}
%
and so
%
\begin{equation*}
    y - x = e^{-t}
\end{equation*}
%
and thus $|y(t) - x(t)| = |e^{-t}| = e^{-t} \to 0$ as $t \to \infty$.

\newpage

7. Let $f \in C(\R^+ \times \R^n, \R^n)$ and $g \in C(\R^+ \times \R, \R^+)$.
Assume that (i) $g(t, u)$ is nondecreasing in $u$ for each fixed $t$ and
%
\begin{equation*}
    |f(t, x)| \leq g(t, |x|), \quad (t, x) \in \R^+ \times \R^n;
\end{equation*}
%
(ii) every solution $u(t) = u(t, t_0, u_0)$ of the IVP
%
\begin{equation*}
    u^\prime(t) = g(t, u(t)), \quad u(t_0) = u_0
\end{equation*}
%
is bounded.

Show that every solution of the IVP
%
\begin{equation*}
    x^\prime(t) = f(t, x(t)), \quad x(t_0) = x_0
\end{equation*}
%
exists on $[t_0, \infty)$ and converges to a constant vector as $t \to \infty$.

\textit{Solution.}
Let $\gamma(t)$ be a maximal solution of
%
\begin{equation*}
    u^\prime(t) = g(t, u(t)), \quad u(t_0) = u_0
\end{equation*}
%
where $u_0 \geq |x_0|$
existing on $[t_0, \infty)$. Then according to Corollary $3.2$ of the course notes,
we have that for any solution $x(t)$ of
%
\begin{equation*}
    x^\prime(t) = f(t, x(t)), \quad x(t_0) = x_0
    ,
\end{equation*}
%
that
%
\begin{equation*}
    |x(t)| \leq \gamma(t) < \infty
\end{equation*}
%
and so $x(t)$ exists on $[t_0, \infty)$. In order for $g(t, u)$ to be nondecreasing in $u$
for any fixed $t$ but still have a bounded solution, its magnitude must be decreasing in $t$ for all
$t \geq t^*$ for some $t^* > t_0$; otherwise, solutions would grow without bound.
Hence we must have that $|g(t, u)| \to 0$ as $t \to \infty$ and thus $|f(t, x)| \to 0$
as $t \to \infty$. Therefore $x(t)$ must converge to a constant vector as $t \to \infty$.

\end{document}
