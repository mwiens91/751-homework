% Set up the document
\documentclass{article}

% Page size
\usepackage[
    letterpaper,]{geometry}
\usepackage{changepage}

% Lines between paragraphs
\setlength{\parskip}{\baselineskip}
\setlength{\parindent}{0pt}

% Math
\usepackage{mathtools}
\usepackage{amssymb}
\usepackage{amsthm}
\usepackage{commath}

% Number sets
\newcommand{\C}{\mathbb{C}}
\newcommand{\N}{\mathbb{N}}
\newcommand{\Q}{\mathbb{Q}}
\newcommand{\R}{\mathbb{R}}
\newcommand{\Z}{\mathbb{Z}}

% Links
\usepackage{hyperref}

% Page numbers at top right
\usepackage{fancyhdr}
\pagestyle{fancy}
\fancyhf{}
\fancyhead[R]{\thepage}
\renewcommand\headrulewidth{0pt}

\begin{document}

\textbf{AMATH 751 assignment 1} \\
\textbf{Matt Wiens} \\
\textbf{2020-08-28}

1. Show that if $\phi(t)$ is a solution of the IVP
%
\begin{equation}
    x^\prime = f(x), \quad x(t_0) = x_0
    \label{eq:1-1}
\end{equation}
%
defined on $\R$, then $\phi(t + t_0)$ is a solution of the IVP
%
\begin{equation}
    x^\prime = f(x), \quad x(0) = x_0
    \label{eq:1-2}
\end{equation}
%
on $\R$. Is it still true if $f(x)$ is replaced by $f(t, x)$?

\textit{Solution.}
Let $\phi$ be a solution of~\eqref{eq:1-1}. Define
$\tilde{\phi}(t) \coloneqq \phi(t + t_0)$. Then
%
\begin{equation}
    \tilde{\phi}(0) = \phi(0 + t_0) = \phi(t_0) = x_0
    \label{eq:1-3}
\end{equation}
%
and
%
\begin{equation}
    \tilde{\phi}^\prime(t) = \phi^\prime(t + t_0) = f(\phi(t + t_0)) = f(\tilde{\phi}(t))
    \label{eq:1-4}
\end{equation}
%
(the first equality comes from a trivial application of the chain rule). Hence
we can conclude that $\tilde{\phi}$ satisfies~\eqref{eq:1-2}.

However, if $f$ is dependent on $t$, then~\eqref{eq:1-3} would still hold but
for~\eqref{eq:1-4} we would have, in general,
%
\begin{equation*}
    \tilde{\phi}^\prime(t)
        = \phi^\prime(t + t_0)
        = f(t + t_0, \phi(t + t_0))
        = f(t + t_0, \tilde{\phi}(t))
        \neq f(t, \tilde{\phi}(t))
        ,
\end{equation*}
%
so $\tilde{\phi}$ would no longer solve the IVP.

\newpage

2. Show that each of the functions
%
\begin{align*}
    |x|_1 &= \sum_{i = 1}^n |x_i|, \\
    |x|_2 &= \sqrt{\sum_{i = 1}^n x_i^2}, \\
    |x|_\infty &= \max_{1 \leq i \leq n} |x_i|
\end{align*}
%
defines a norm on $\R^n$.

\textit{Solution.}
The first property that norms $N$ must satisfy is that $N(x) \geq 0$ and
$N(x) = 0$ if and only if $x = 0$. Quite trivially, we can see
that $|\cdot|_1$, $|\cdot|_2$, and $|\cdot|_\infty$ are all non-negative
and that $|0|_1 = |0|_2 = |0|_\infty = 0$. Now suppose $x \neq 0$ where $x \in \R^n$,
and let $j$ be such that $x_j \neq 0$. Then we have
%
\begin{align*}
    |x|_1 &= \sum_{i = 1}^n |x_i| \geq |x_j| > 0, \\
    |x|_2 &= \sqrt{\sum_{i = 1}^n x_i^2} \geq \sqrt{x_j^2} > 0, \\
    |x|_\infty &= \max_{1 \leq i \leq n} |x_i| \geq |x_j| > 0.
\end{align*}
%
Hence for all three functions $|\cdot|$, we have that $|x| = 0$ if and only if $x = 0$.
Therefore all thee functions satisfy the first property in the definition of a norm.

The second property that norms must satisfy is that $N(x + y) \leq N(x) + N(y)$.
Let $x, y \in \R^n$, then we have that, using the triangle inequality,
%
\begin{align*}
    |x + y|_1
        &= \sum_{i = 1}^n |x_i + y_i| \\
        &\leq \sum_{i = 1}^n (|x_i| + |y_i|) \\
        &= \sum_{i = 1}^n |x_i| + \sum_{i = 1}^n |y_i| \\
        &= |x|_1 + |y|_1
        .
\end{align*}
%
We also have that, using the Cauchy-Schwarz inequality,
%
\begin{align*}
    |x + y|_2^2
        &= \sum_{i = 1}^n (x_i + y_i)^2 \\
        &= \sum_{i = 1}^n (x_i^2 + 2 x_i y_i +  y_i^2) \\
        &= \sum_{i = 1}^n x_i^2
            + \sum_{i = 1}^n  y_i^2
            + 2 \sum_{i = 1}^n x_i y_i \\
        &= |x|_2^2 + |y|_2^2 + 2 \langle x, y \rangle \\
        &\leq |x|_2^2 + |y|_2^2 + 2 |x|_2 |y|_2 \\
        &= (|x|_2 + |y|_2)^2
        .
\end{align*}
%
Taking square roots (and remembering that we showed that $|\cdot|_2$
is non-negative above), we obtain
%
\begin{equation*}
    |x + y|_2 \leq |x|_2 + |y|_2
    .
\end{equation*}
%
Additionally, have that
%
\begin{align*}
    |x + y|_\infty
        &= \max_{1 \leq i \leq n} |x_i + y_i| \\
        &= \max_{1 \leq i \leq n} |x_i|
            + \max_{1 \leq i \leq n} |y_i| \\
        &= |x|_\infty + |y|_\infty
        .
\end{align*}
%
Hence all three functions satisfy the second property in the definition of a norm.

Finally, the third property that norms must satisfy is that for all $\alpha \in \R$,
$N(\alpha x) = |\alpha| N(x)$. Let $x \in \R^n$ and $\alpha \in \R$, then we have that
%
\begin{align*}
    |\alpha x|_1
        &= \sum_{i = 1}^n |\alpha x_i|
        = \sum_{i = 1}^n |\alpha| |x_i|
        = |\alpha| \sum_{i = 1}^n |x_i|
        = |\alpha| |x|_1
        , \\
    |\alpha x|_2
        &= \sqrt{\sum_{i = 1}^n (\alpha x_i)^2}
        = \sqrt{\sum_{i = 1}^n \alpha^2 x_i^2}
        = \sqrt{\alpha^2 \sum_{i = 1}^n x_i^2}
        = \sqrt{\alpha^2} \sqrt{\sum_{i = 1}^n x_i^2}
        = |\alpha| \sqrt{\sum_{i = 1}^n x_i^2}
        = |\alpha| |x|_2
        , \\
    |\alpha x|_\infty
        &= \max_{1 \leq i \leq n} |\alpha x_i|
        = \max_{1 \leq i \leq n} |\alpha| |x_i|
        = |\alpha| \max_{1 \leq i \leq n} |x_i|
        = |\alpha| |x|_\infty
        .
\end{align*}
%
Hence all three functions satisfy the third (and final) property in the definition of a norm.

Thus we have shown that $|\cdot|_1$, $|\cdot|_2$, and $|\cdot|_\infty$ are norms.

\newpage

3. Show that if $g_m \in C^1([a, b], \R^n)$ for all $m \in \N$ and $\{g_m^\prime(t)\}$
is uniformly bounded on $[a, b]$, then $\{g_m(t)\}$ is equicontinuous on $[a, b]$.

\textit{Solution.}
hey

\newpage

4. Determine if the following integral equation has a unique solution for some $\alpha > 0$:
%
\begin{equation*}
    x(t) = 2 + \int_0^t \frac{1}{3 + s^2} \sqrt{2 + x(s)^2} \dif s, \quad t \in [-\alpha, \alpha]
    .
\end{equation*}

\textit{Solution.}
hey

\newpage

5. Determine if the following IVP has a unique solution where $x(t) = (x_1(t), x_2(t))$:
%
\begin{align*}
    x_1^\prime(t) &= x_1(t) \sin(t) - 2 x_1(t)^2 - x_1(t) x_2(t), \\
    x_2^\prime(t) &= 2 t^2 x_2(t) - x_2(t)^2 - x_1(t) x_2(t), \\
    x(0) &= x_0.
\end{align*}

\textit{Solution.}
hey

\newpage

6. Let $f \in C(Q, \R^n)$, where $Q \subset \R^{n + 1}$ is an open set. Assume
that $f(t, x)$ has continuous partial derivatives with respect to $x$. Show that
$f(t, x)$ is locally Lipschitz in $x$ in $Q$, and Lipshitz in $x$ in $x$ on any compact
and convex subset of $Q$.

\textit{Solution.}
hey

\newpage

7. Let $f: \R^n \to \R^n$ be continuously differentiable. Show that two distinct solutions
of the autonomous system
%
\begin{equation*}
    x^\prime = f(x)
\end{equation*}
%
cannot intersect at a point, not even at different times.

\textit{Solution.}
hey

\end{document}
